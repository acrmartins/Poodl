\documentclass{article}
%\documentclass[twocolumn,showpacs,showkeys,preprintnumbers,amsmath,amssymb]{revtex4}
\usepackage[american]{babel}
\usepackage[latin1]{inputenc}
\usepackage{graphicx}
\usepackage{xcolor}
\usepackage[colorinlistoftodos]{todonotes}



\begin{document}


\title{Some title}

%\vspace{1cm}

\author{Marcelo Maciel \and Andr\'e C. R. Martins\\
	%}
	%\email{amartins@usp.br}
	%\affiliation{%
	NISC - EACH, Universidade de S\~ao Paulo\\
	Av. Arlindo B\'etio, 1000, S\~ao Paulo, 03828-080, Brazil}

%\author{Andr\'e C. R. Martins\\
	%}
 %\email{amartins@usp.br}
 %\affiliation{%
%NISC - EACH, Universidade de S\~ao Paulo\\
%Av. Arlindo B\'etio, 1000, S\~ao Paulo, 03828-080, Brazil}
 

\date{}


\maketitle
%\vspace{0.3cm}

%\center{\Large\bf Segundo autor}
%\center{\large\bf Afilia��o}


\begin{abstract}
	
	
	%\keywords{}
	%\pacs{}
\end{abstract}


\section{Introduction}

 % Incluir aqui a parte sobre teoria pol�tica, um par�grafo � suficiente, a princ�pio.

In this paper, we will propose a opinion dynamics model
\cite{castellanoetal07,galam12a,galametal82,galammoscovici91,sznajd00,deffuantetal00,martins08a}
to explore the consequences of the existence of issues that can be interpreted
as opinions over a one-dimensional axis. \textcolor{blue}{It's usual to think
  about policy alternatives and agents' preferences spatially (geometrically),
  that is, through a mapping from similarity to proximity
  \cite{downs1957economic, laver2014measuring}. The model then captures the
  daily notion of parties or policies being more ``to the left'' or ``right''
  than others, that is, if they're similar then they're closer
  \cite{van2005political, miller2015spatial}.} Major opinions, including
political ones, tend to be formed from how each person feels about a number of
issues. Locating someone in a left versus right or liberal versus conservative
axis, \textcolor{blue}{ therefore}, requires inspecting the opinions of that
person in not only one but a number of different issues that constitute the
ideological positioning \cite{benoit2006party}.

While it would make sense to consider different issues as having components in
more than one single dimension \cite{vicenteetal08b}, looking at the problem as
one-dimensional can be justified in several ways. We can certainly see this as a
first approximation along the most relevant dimension. In this case, we are
simply investigating the projection of higher-dimensional problems along a
direction where variation seems especially important. And, from the point of
view of applications, it is usual to find discussions to be simplified over a
main disagreement. \textcolor{blue}{ Even though there are many variants of this
  modeling strategy, for our work what matters is that this naturally leads to
  the use of continuous opinion models} such as the Bounded Confidence (BC)
models \cite{deffuantetal00,hegselmannkrause02}. While discrete models
\cite{galametal82,galammoscovici91,sznajd00} can be very useful at describing
choices, they are not easiest way to represent strength of opinion. Discrete
models also do not naturally provide a scale where we can compare opinions and
decide which one is more to the right or more liberal.

On the other hand, continuous models are not particularly well suited for
problems involving discrete decisions. \textcolor{red}{As we will not deal with
  those kinds of problems here, they are a natural choice.} \textcolor{blue}{
  (\(\leftarrow\) i dont understand this sentence; prof, could u explain?? )}
Indeed, continuous opinions models have been proposed for several different
problems on how opinions spread on a society
\cite{deffuantetal02a,weisbuchetal05}, from questions about the spread of
extremism
\cite{amblarddeffuant04,gargiulomazzoni08a,franksetal08a,alizadeh14a,Albi2016,Mai2017}
to other issues such as how different networks
\cite{Kurmyshev2011,Acemoglu2011,Das2014,Hu2017} or the uncertainty of each
agent \cite{deffuant06} might change how agents influence each other.

Here, we will use a continuous opinion model created by Bayesian-like reasoning
\cite{martins08c}, inspired by the Continuous Opinions and Discrete Actions
(CODA) model \cite{martins08a,martins12b}. The model was shown previously
\cite{martins08c} to provide the same qualitative results as BC models. While a
little less simple, the Bayesian basis make for a more clear interpretation of
the meaning of the variables, as we extend the model and need to interpret the
new results, \textcolor{blue}{and is consistent with a boundedly rational
  variant interpretation of the spatial model of political decision making
  \cite{humphreys2010spatial,ostrom1998behavioral}.}


% % % Explicar aqui em termos bem gerais o modelo

We will also study a variation of our model where the function of trust $p^*$
will not be influenced by the distance between the opinions of the agent and the
neighbor on the specific issue they are debating. Instead, $p^*$ will be
determined by the distance between the neighbor opinion and the average opinion
of the agent. The idea here is to make the behavior of our agents closer to what
experiments show about human reasoning. We have observed that our reasoning
about political problems can be better described as ideologically motivated
\cite{jostetal03a,taberlodge06a,Claassen2015a}. Indeed, our opinions tend to
come in blocks even when the issues are logically independent \cite{jervis76a}.
Our reasoning abilities seem to exist more to defend our main point of views
\cite{mercier11a,merciersperber11a} and our cultural identity \cite{kahanetal11}
than to find the best answer. In that context, evaluating other by how they
differ from us as a whole, instead of in each issue, is a model variation worth
exploring.
 






\section{The Model}

\textcolor{blue!60}{
The model is an agent-based social simulation \cite{de2014agent}. At the initial
condition of the simulation we have a population of \(N\) agents which have an
ideological profile
\(I_i
=
(
(o_{i, 1}, \sigma),
\ldots,
(o_{i, n}, \sigma)
)
\)
, where \( 1 \leq n \leq 10 \) is the number of issues, \( 0.0 < o < 1.0 \) is
the opinion about the issue and \( 0.01 \leq \sigma \leq 0.5\) is a global
variable which can be interpreted as the uncertainty about the issue
\cite{martins12b}. Another attribute is the agent's ideological position at the
dimension of interest, or ideal point \cite{armstrong2014analyzing}, which we
treat as the arithmetic mean of its opinions in each issue
\(
x_i
=
\frac{1}{n}
\sum_{k=1}^{n}
o_{k}
\).
}

\textcolor{blue!60}{
The initial \(o_i\)s for each issue are sampled from Beta\((\alpha, \beta)\)
distributions where each agent is associated with its own pair
\( (
\alpha
\in [1.1, 100],
\beta
\in [1.1, 100]
)
\)
.
The reason for this is that if we sample the \(o\)s from an Uniform distribution
as we increase the number of issues (\(n\)) the closer to the center of the
dimension the agents' ideological position (\(x\)) would be.
\todo[color=yellow!40]{maybe refactor this sentence?} Using a Beta distribution
prevents this, lets the initial \(o\)s of each agent to be correlated, since
they're drawn from the agent's own Beta, and lets us have an initial population
of agents with ideological positions distributed along the dimension, instead of
clustered around the center.
}

\textcolor{blue!60}{
  For its part, \(\sigma\) is a global variable, that is, a
  parameter of the model. A certain proportion of the agents will have an unique
  \(\sigma_{i,k} = 1e-20\), so that we can control for the impact of
  \textit{intransigent} agents on the model dynamics \cite{deffuant2002can}. How
  many agents are intransigent is also a parameter (coded as
  \(
  0.0
  \leq
  \textit{p$\_$intran}
  \leq
  0.1
  \)
  ), and such \(\sigma\) is established at the initial
  condition by sampling the issue index from the \(I_i\)'s length.}

\textcolor{blue!60}{ An iteration of the simulation is the application of two
  procedures: the opinion update through social influence and a random opinion
  update (noise). In the social influence procedure we draw a single agent \(i\)
  from the population. We then draw another agent \(j\) from the population. We
  then draw one of the issues \(k \in (1 , \ldots, n)\) so that we have the
  corresponding pairs (\(o_{i,k}, o_{j,k}\)) and (\(\sigma_{i,k},
  \sigma_{j,k}\)). The agent \(i\) then updates its opinion (\(o_{i,k}\))
  following the equation
  \[
    o_{i,k}(t+1) =
    p^{*}
    \frac{o_{i,k}(t) + o_{j,k}(t) }{2}
    +
    (1 - p^{*})
    o_{i,k}(t).
  \]
  Wherein 
  \[
   p^{*}
    =
  \frac{
      p \frac{1}{\sqrt{2 \pi} \sigma_i}
      e^{- (\frac{ ( o_i (t) - o_j (t))^2}{2 \sigma_i^2})}
    }{
      p
      \frac{1}{\sqrt{2 \pi} \sigma_i}
    e^{- (\frac{ ( o_i (t) - o_j (t))^2}{2 \sigma_i^2})}
    +
    (1 - p)
  }.
\] In which \(p\) is a global parameter used to model the likelihood of the
other agent's (\(j\)) opinion being true \cite{martins12b}. Furthermore, we have
the noise, in which we draw another agent \(i\) whose opinion \( o_{i,k}(t+1)\)
is equal to \(o_{i,k}(t) + r\) where \(r\) is taken from a Normal distribution
of mean 0 and standard deviation \(\rho\). \(\rho\) is then a global parameter
of the simulation. From a theoretical point of view the noise is justified as a
way of accounting for the effect of factors not related to social influence that
make the agents change their opinion about issues \cite{flache2017}. A further
methodological justification is that small perturbations in the local behavior
of agents may lead to drastic changes in systemic properties
\cite{macy2015signal}. If an agent \(i\) is intransigent in an issue \(k\) it
won't randomly change its \(o_{i,k}\) opinion if its chosen by the noise
algorithm.
}

\section{Conclusions}






\section{Acknowledgement}
The author would like to thank Funda\c{c}\~ao de Amparo \`a Pesquisa do Estado de S\~ao Paulo (FAPESP), for the support to this work, under grant %2008/00383-9.

\bibliographystyle{unsrt}
\bibliography{biblio}

\end{document}

%%% Local Variables:
%%% mode: latex
%%% TeX-master: t
%%% End:
